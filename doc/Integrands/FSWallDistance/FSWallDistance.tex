\documentclass[twoside, 11pt, a4paper]{article}
\usepackage{amsmath,amsfonts,amssymb,graphicx,parskip}
\usepackage[utf8]{inputenc}

\DeclareMathOperator{\eps}{\epsilon}
\newcommand{\dee}{\mathrm{d}}

\title{A differential equation for approximate wall distance - SplineFEM implementation}
\author{Arne Morten Kvarving}
\begin{document}
\maketitle
This document describe the derivation of the weak form, and the associated
Jacobian for the equation given for approximate wall distance in
\cite{fares}. It is implemented in SplineFEM in the FSWallDistance
Integrand, see src/Integrands/FSWallDistance.h/C, while
you can find a test application (and SIM classes) in src/Apps/FSWallDistance.

The equation is expressed for the inverse wall distance $\mathcal{G}$.
This field should be initialized to $\mathcal{G}_{\text{init}} \neq 0$ (any
constant value),
except on the walls where it should be initialized to
$\mathcal{G}_{\text{wall}} = \frac{1}{\mathcal{D}_0} = \mathcal{G}_0$.
The value can be used to simulate rough walls. In any case we must put
$\mathcal{D}_0 > 0$.

Initial expression:
\begin{equation}
	\left(\mathcal{G}\mathcal{G}_x\right)_x + \left(\mathcal{G}\mathcal{G}_y\right)_y+\left(\mathcal{G}\mathcal{G}_z\right)_z + \left(\sigma-1\right)\mathcal{G}\left(\mathcal{G}_{xx}+\mathcal{G}_{yy}+\mathcal{G}_{zz}\right)-\gamma\mathcal{G}^4=0
\end{equation}
Using vector notation this is:
\begin{equation}
	\nabla\cdot\left(\mathcal{G}\nabla \mathcal{G}\right) + \left(\sigma-1\right)\mathcal{G}\nabla^2\mathcal{G} - \gamma\mathcal{G}^4
\end{equation}
Multiply with test function, perform integration by parts:
\begin{eqnarray}
	\int_\Omega \nabla\cdot\left(\mathcal{G}\nabla{G}\right)v\,\dee\Omega + \left(\sigma-1\right)\int_\Omega\mathcal{G}\nabla^2\mathcal{G}v\,\dee\Omega - \gamma\int_\Omega\mathcal{G}^4v\,\dee\Omega = \\
	\int_{\partial\Omega}\mathcal{G}\nabla\mathcal{G}v\cdot \mathbf{n}\,\dee S - \int_\Omega\mathcal{G}\nabla\mathcal{G}\cdot\nabla v\,\dee\Omega +\,  \\
	\left(\sigma-1\right)\left(\int_{\partial\Omega}\mathcal{G}\cdot\nabla\mathcal{G}v\cdot\mathbf{n}\,\dee S - \int_\Omega \mathcal{G}\nabla\mathcal{G}\cdot\nabla v\,\dee\Omega-\int_\Omega\nabla^2Gv\,\dee\Omega\right)- \\
	\gamma\int_\Omega\mathcal{G}^4v\,\dee\Omega = 0.
\end{eqnarray}

As far as the Jacobian goes, we are looking for
\begin{equation}
	\frac{\dee}{\dee \mathcal{G}}\left(\nabla \cdot \left(\mathcal{G}\nabla\mathcal{G}\right)+\left(\sigma-1\right)\mathcal{G}\nabla^2\mathcal{G}-\gamma\mathcal{G}^4\right).
\end{equation}
We ``cheat'' and insert a delta value;
\begin{equation}
	\begin{aligned}
		\nabla\cdot\left(\left(\mathcal{G}+\delta \mathcal{G}\right)\nabla\left(\mathcal{G}+\delta \mathcal{G}\right)\right) + \left(\sigma-1\right)\left(\mathcal{G}+\delta \mathcal{G}\right)\nabla^2\left(\mathcal{G}+\delta \mathcal{G}\right)-\gamma\left(\mathcal{G}+\delta \mathcal{G}\right)^4 \Rightarrow \\
		\nabla\cdot\left(\mathcal{G}\nabla \mathcal{G}+\mathcal{G}\nabla\delta\mathcal{G}+\delta\mathcal{G}\nabla\mathcal{G}+\delta\mathcal{G}\nabla\delta\mathcal{G}\right) + \\
		\left(\sigma-1\right)\left(\mathcal{G}\nabla^2\mathcal{G}+\mathcal{G}\nabla^2\delta\mathcal{G}+\delta\mathcal{G}\nabla^2\mathcal{G} + \delta\mathcal{G}\nabla^2\delta\mathcal{G}\right) - \\
		\gamma\left(\mathcal{G}^4+4\mathcal{G}^3\delta\mathcal{G}+6\mathcal{G}^2\delta\mathcal{G}^2 + 4\mathcal{G}\delta\mathcal{G}^3+\delta\mathcal{G}^4\right).
	\end{aligned}
\end{equation}
Linearizing, we end up with
\begin{equation}
	\nabla\cdot\left(\mathcal{G}\nabla\delta\mathcal{G}+\delta\mathcal{G}\nabla\mathcal{G}\right) + \\
	\left(\sigma-1\right)\left(\mathcal{G}\nabla^2\delta\mathcal{G}+\delta\mathcal{G}\nabla^2\mathcal{G}\right) - \\
	\gamma\left(4\mathcal{G}^3\delta\mathcal{G}\right) = 0.
\end{equation}
Weak form of the Jacobian, term by term:
\begin{equation}
	\begin{aligned}
		\int_\Omega \nabla\cdot\left(\mathcal{G}\nabla\delta\mathcal{G}+\delta\mathcal{G}\nabla\mathcal{G}\right)\,\dee\Omega &=&
		\int_{\partial\Omega} \mathcal{G}\nabla\delta\mathcal{G}v\cdot\mathbf{n}\,\dee S - \int_\Omega\mathcal{G}\nabla\delta\mathcal{G}\cdot\nabla v\,\dee\Omega= \\
		&+&\int_{\partial\Omega} \delta\mathcal{G}\nabla\mathcal{G}v\cdot\mathbf{n}\,\dee S - \int_\Omega\delta\mathcal{G}\nabla\mathcal{G}\cdot\nabla v\,\dee\Omega
	\end{aligned}
\end{equation}
\begin{equation}
	\begin{aligned}
		\int_\Omega\left(\mathcal{G}\nabla^2\delta\mathcal{G}+\delta\mathcal{G}\nabla^2\mathcal{G}\right)v\,\dee\Omega &=&
		\int_{\partial\Omega}\mathcal{G}\nabla\delta\mathcal{G}v\,\dee S - \int_\Omega\nabla\mathcal{G}\cdot\nabla\left(\delta\mathcal{G}v\right)\,\dee\Omega \\
		&+&\int_{\partial\Omega}\delta\mathcal{G}\nabla\mathcal{G}v\,\dee S - \int_\Omega\nabla\delta\mathcal{G}\cdot\nabla\left(\mathcal{G}v\right)\,\dee\Omega \\
		&=& -\int_\Omega\nabla\mathcal{G}\cdot\nabla\delta\mathcal{G}v\,\dee\Omega - \int_\Omega\delta\mathcal{G}\nabla\mathcal{G}\cdot\nabla v\,\dee\Omega \\
		& & -\int_\Omega\nabla\delta\mathcal{G}\cdot\nabla\mathcal{G}v\,\dee\Omega - \int_\Omega\mathcal{G}\nabla\delta\mathcal{G}\cdot\nabla v\,\dee\Omega
	\end{aligned}
\end{equation}
The last term is straight forward;
\begin{equation}
	\int_\Omega 4\mathcal{G}^3\delta\mathcal{G}v\,\dee\Omega
\end{equation}

In summary, our Jacobian reads
\begin{equation}
	\begin{aligned}
		&-&\int_\Omega\mathcal{G}\nabla\delta\mathcal{G}\cdot\nabla v\,\dee \Omega - \int_\Omega\delta\mathcal{G}\nabla\mathcal{G}\cdot\nabla v\,\dee\Omega \\
		&-&\left(\sigma-1\right)\left(\int_\Omega\nabla\mathcal{G}\cdot\nabla\delta\mathcal{G}v\,\dee\Omega - \int_\Omega\delta\mathcal{G}\nabla\mathcal{G}\cdot\nabla v\,\dee\Omega\right) \\
		&-&\left(\sigma-1\right)\left(\int_\Omega\nabla\delta\mathcal{G}\cdot\nabla\mathcal{G}v\,\dee\Omega - \int_\Omega\mathcal{G}\nabla\delta\mathcal{G}\cdot\nabla v\,\dee\Omega\right) \\
		&-&\gamma\int_\Omega 4\mathcal{G}^3\delta\mathcal{G}v\,\dee\Omega
	\end{aligned}
\end{equation}
where
\[
	\sigma \geq 0.2, \gamma = 1+2\sigma.
\]

\bibliographystyle{plain}
\bibliography{references}

\end{document}

